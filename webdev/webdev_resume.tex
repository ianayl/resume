%
% TODO figure out formatting of resume, it sucks rn
%

\documentclass[9pt]{article}

\usepackage{enumitem}
\usepackage{titlesec}
\usepackage{titling}
\usepackage{mathptmx}
\usepackage{hyperref}
\usepackage[margin=0.5in]{geometry}
% \usepackage{fullpage}

% I don't want indents
\setlength{\parindent}{0in}

\pagenumbering{gobble}

% Cool link colors
\hypersetup{
    colorlinks=true,
    linkcolor=blue,
    urlcolor=blue,
}

\titleformat{\section}
{\Large\bfseries}
{}
{0em}
{}[\titlerule]

\titleformat{\subsection}
{\bfseries\large}
{}
{0em}
{}

\titleformat{\subsubsection}[runin]
{\bfseries}
{}
{0em}
{}[  --]

% \renewcommand{\familydefault}{\sfdefault}

\renewcommand{\maketitle}{
		\vspace{-8em}
    \begin{center}
		{\huge\bfseries
		\theauthor}
    \vspace{0.5em}
    \\
    % \textit{Address Redacted} \\
    % \textit{Email Redacted} --- \textit{Phone Redacted} \\
    Github: \href{https://github.com/ianayl}{@ianayl}, Linkedin: \href{https://www.linkedin.com/in/ianayl}{ianayl}, Email: \url{ianayl.work@gmail.com}
    \end{center}
    \vspace{-1.5em}
}

\titlespacing{\subsubsection}
{0em}{0.25em}{0.5em}

\titlespacing{\subsection}
{0em}{1.5em}{0em}

\setitemize{noitemsep}

\author{Ian Li}

\begin{document}

\maketitle

% \section{Objective}
% \vspace{0.1em}
% \begin{itemize}
% 	\item To obtain a co-op position at \textit{whatever}\\
% 	\vspace{-1.5em}
% 	\item To gain exposure to new techniques and technologies, and to obtain new skills
% \end{itemize}

\section{Skills}

\begin{tabular}{ l l }
    \textbf{Programming Languages} & Javascript/Typescript, Python, C++, CUDA, C, C\#, Java, Bourne Sh/Bash, Scheme, ARM assembly, R \\
    \textbf{Tooling} & Linux, MS Azure, Selenium, LLVM, Nix/NixOS, Regex (PCRE), Make, Git, SSH, GDB/LLDB \\
    \textbf{Web Development} & React, React Native, .NET Core, SQL, Express, SASS, Bootstrap, MongoDB, Flask \\
    \textbf{Certificates} & Microsoft Azure AZ-900, Microsoft Azure AI-900
\end{tabular}

\section*{Experiences}

% \subsection{Software Developer, \textit{FIRST Robotics Team 6866} - \normalsize\textit{Markham, Canada}\hfill \normalsize\textnormal{2020}}
% \begin{itemize}
%   \item Developed a control system and API for a conveyor magazine system to shoot at targets in \textbf{Java} using WPILib
%   \item Contributed to custom PID controller class for precise control of motors
% \end{itemize}

\subsection{Web Developer, \textit{nvision} - \normalsize\textit{Markham, Canada} \hfill \normalsize\textnormal{2019}}
\begin{itemize}
  \item Independently developed employee dashboards for visualization of metrics using \textbf{React} and \textbf{MongoDB} (\textbf{MERN} stack)
  \item Developed customizable Wordpress themes, collaborating with UI designers to bring design mockups into fruition
  \item Maintained \textbf{Linux} servers and MySQL databases to host Wordpress sites using the \textbf{LAMP} stack
\end{itemize}

\subsection{Lead Software Developer, \textit{Rentura (startup)} - \normalsize\textit{Toronto, Canada} \hfill \normalsize\textnormal{2021-2022}}
Lead the development and design of a \textit{minimum viable product} for a B2C furniture rental platform
\vspace{-0.5em}
\begin{itemize}
  \item Independently developed server backend in \textbf{Javascript}, with a \textbf{RESTful API} using \textbf{CRUD} endpoints secured with \textit{tokens}
  \item Maintained \textbf{MongoDB} instances on a \textbf{Linux} server, and designed database schemas for an \textit{order management system}
  \item Collaboratively developed responsive, mobile-first \textit{e-commerce} frontends in \textbf{React.js} using \textbf{Sass} and \textbf{Axios} (\textbf{MERN} stack)
    % \item TODO add business stuff later
    % \item 2021\--Current
\end{itemize}

% \subsection{System Administrator, \textit{Gang RP Network} \hfill \normalsize\textnormal{2017-2018}}
% \begin{itemize}
%     \item Created custom game mods and 3D assets for game servers using Unity and Blender
%     \item Maintained servers through Telnet, maintained internal management dashboards and image assets
% \end{itemize}

% \subsection{Computer Repairs} 
% \subsubsection{Home} Repaired laptop components, reflashed BIOS, fixed and reinstalled Operating Systems for clients privately 

% \subsection{Teaching Assistant}
% \subsubsection{Markham}
% Ran Google CS/STEM workshops (hour of code, scratch, basic 3D modelling, etc) --- 2014\--2018  

\section*{Projects}

% \subsection{Rentura \textnormal{(Startup)} \hfill \normalsize\textnormal{2021 -- 2022}}
% Lead the development and design of a \textit{minimum viable product} for a B2C furniture rental platform
% \vspace{-0.5em}
% \begin{itemize}
%   \item Independently developed server backend in \textbf{Javascript}, with a \textbf{RESTful API} using \textbf{CRUD} endpoints secured with \textit{tokens}
%   \item Maintained \textbf{MongoDB} instances on a \textbf{Linux} server, and designed database schemas for an \textit{order management system}
%   \item Collaboratively developed responsive, mobile-first \textit{e-commerce} frontends in \textbf{React.js} using \textbf{Sass} and \textbf{Axios} (\textbf{MERN} stack)
%     % \item TODO add business stuff later
%     % \item 2021\--Current
% \end{itemize}


\subsection{Mindless Password Manager \normalsize\textnormal{- \href{https://github.com/ianayl/mindless}{repo}}\hfill \normalsize\textnormal{2021}}
    Proof-of-concept \textbf{unconditionally secure} biometrics password manager based on hashing algorithms (SHA256)
    \vspace{-0.5em}
\begin{itemize}
  \item Developed a password manager using facial recognition landmarks (extracted using \textbf{OpenCV}) with a web frontend (\textbf{Flask})
  \item Created a method to generate consistent passwords from facial landmark data: \textit{passwords are \textbf{never} stored! Not even encrypted}
    \item Became a finalist project in a 36-hour hackathon (Hackwestern 8) as a one-man team out of 346 teams
\end{itemize}

\subsection{Experimental Language \normalsize\textnormal{- \href{https://github.com/ianayl/compiler}{repo}}\hfill \normalsize\textnormal{Current}}
Shells are \textit{outdated}: Creating a language in \textbf{C++} with goals to be a good general-purpose language \textit{and} a fast replacement to shells
\vspace{-0.5em}
\begin{itemize}
  \item Devised an automatic table-based \textbf{lexer generator}, and wrote a \textbf{recursive descent} parser following LL grammar
  \item Writing an interpreter as a \textbf{tree-walk evaluator}, and AST passes to generate CFGs / \textbf{LLVM IR} for compilation
  % \item Working on rewrite using a \textbf{LR} parser with integrated lexer, a compiler to \textbf{LLVM} IR, and a strong typing system
        % TODO phrasing here sucks balls redo it
  \item Working on a static type system with type inference, goals for JIT compilation in future using LLVM
\end{itemize}


\subsection{Shell Site Generator (\texttt{shsg}) \normalsize\textnormal{- \href{https://github.com/ianayl/shsg}{repo}}\hfill \normalsize\textnormal{Current \textit{(On hiatus)}}}
  Ultra lightweight and portable static site generator that will run on \textit{anything} with a minimal UNIX-like environment
    % Easy to use static site generator that can be deployed on any minimal UNIX-like environment
\vspace{-0.5em}
\begin{itemize}
  \item Developing a static site generator using \textbf{POSIX shell} and \textbf{PCRE Regex}: \textit{No dependencies other than a UNIX shell \& coreutils!}
  \item Created a transpiler, transpiling markdown to themed webpages with no installation, compilation, or alternative runtimes required
    \item Working on removing dependency on PCRE extensions (support for Busybox), and parsing tables and lists in pure regex
\end{itemize}

\section*{Ongoing Thesis}

\subsection{Improving KLARAPTOR: Automatic Finding of Optimal CUDA Kernel Launch Parameters\hfill \normalsize\textnormal{Current}}
Automatically find near-optimal \textbf{CUDA} kernel parameters for fast GPU kernel execution. I am working on a(n):
\vspace{-0.5em}
\begin{itemize}
  \item Method to detect constraints on kernel launch parameters using Scalar Evolution (SCEV) and Quantifier Elimination

  \item \textbf{LLVM} pass to find range of all valid launch parameters for a given kernel (to facilitate profiling using KLARAPTOR)
  % \item Working on rewrite using a \textbf{LR} parser with integrated lexer, a compiler to \textbf{LLVM} IR, and a strong typing system
        % TODO phrasing here sucks balls redo it
\end{itemize}
\vspace{-0.5em}
In order to speed up \textit{data-oblivious} computations (\textit{e.g.} matrix operations for machine learning/AI, computer graphics, \textit{etc.})


\section{Education}
\subsection{University of Western Ontario - \normalsize\textit{London, Canada} \hfill \normalsize\textnormal{2021 -- Current}}
\subsubsection{Candidate for Honours Specialization in Computer Science} Anticipated in Spring 2025, 3.9/4.0 GPA 

% \subsubsection{Awards} Dean's Honor List (2021-2022, 2022-2023),  % \section{References}
% References available upon request

\end{document}
