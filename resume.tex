%
% TODO figure out formatting of resume, it sucks rn
%

\documentclass[9pt]{article}

\usepackage{enumitem}
\usepackage{titlesec}
\usepackage{titling}
\usepackage{mathptmx}
\usepackage{hyperref}
\usepackage[margin=0.5in]{geometry}
% \usepackage{fullpage}

% I don't want indents
\setlength{\parindent}{0in}

\pagenumbering{gobble}

% Cool link colors
\hypersetup{
    colorlinks=true,
    linkcolor=blue,
    urlcolor=blue,
}

\titleformat{\section}
{\Large\bfseries}
{}
{0em}
{}[\titlerule]

\titleformat{\subsection}
{\bfseries\large}
{}
{0em}
{}

\titleformat{\subsubsection}[runin]
{\bfseries}
{}
{0em}
{}[  --]

% \renewcommand{\familydefault}{\sfdefault}

\renewcommand{\maketitle}{
		\vspace{-7em}
    \begin{center}
		{\huge\bfseries
		\theauthor}
    \vspace{0.5em}
    \\
    % \textit{Address Redacted} \\
    % \textit{Email Redacted} --- \textit{Phone Redacted} \\
    Github: \href{https://github.com/ianayl}{@ianayl}, Linkedin: \href{https://www.linkedin.com/in/ianayl}{ianayl}, Email: \url{ianayl.work@gmail.com}
    \end{center}
    \vspace{-1.0em}
}

\titlespacing{\subsubsection}
{0em}{0.25em}{0.5em}

\titlespacing{\subsection}
{0em}{1.5em}{0em}

\setitemize{noitemsep}

\author{Ian Li}

\begin{document}

\maketitle

% \section{Objective}
% \vspace{0.1em}
% \begin{itemize}
% 	\item To obtain a co-op position at \textit{whatever}\\
% 	\vspace{-1.5em}
% 	\item To gain exposure to new techniques and technologies, and to obtain new skills
% \end{itemize}

\section{Skills}

\begin{tabular}{ l l }
    \textbf{Programming Languages} & C, C++, C\#, Python, Java, Javascript, Typescript, POSIX Sh, Bash, Scheme, ARM assembly, R \\
    \textbf{Tooling} & LLVM, Linux, Nix/NixOS, MS Azure, Selenium, PCRE Regex, Make, Git, SSH, GDB/LLDB \\
    \textbf{Web Development} & React, React Native, .NET Core, SQL, Express, SASS, Bootstrap, MongoDB, Flask \\
    \textbf{Certificates} & Microsoft Azure AZ-900, Microsoft Azure AI-900
\end{tabular}

\section*{Experiences}

% \subsection{Software Developer, \textit{FIRST Robotics Team 6866} - \normalsize\textit{Markham, Canada}\hfill \normalsize\textnormal{2020}}
% \begin{itemize}
%   \item Developed a control system and API for a conveyor magazine system to shoot at targets in \textbf{Java} using WPILib
%   \item Contributed to custom PID controller class for precise control of motors
% \end{itemize}

\subsection{Web Developer, \textit{nvision} - \normalsize\textit{Markham, Canada} \hfill \normalsize\textnormal{2019}}
\begin{itemize}
  \item Developed internal employee dashboards for visualizing metrics using \textbf{React} and \textbf{MongoDB} (\textbf{MERN} stack) and \textbf{Bootstrap} 
  \item Maintained a \textbf{Linux} server to host Wordpress sites using the \textbf{LAMP} stack; Developed customizable Wordpress themes
\end{itemize}

% \subsection{System Administrator, \textit{Gang RP Network} \hfill \normalsize\textnormal{2017-2018}}
% \begin{itemize}
%     \item Created custom game mods and 3D assets for game servers using Unity and Blender
%     \item Maintained servers through Telnet, maintained internal management dashboards and image assets
% \end{itemize}

% \subsection{Computer Repairs} 
% \subsubsection{Home} Repaired laptop components, reflashed BIOS, fixed and reinstalled Operating Systems for clients privately 

% \subsection{Teaching Assistant}
% \subsubsection{Markham}
% Ran Google CS/STEM workshops (hour of code, scratch, basic 3D modelling, etc) --- 2014\--2018  

\section*{Undergraduate Thesis}

\subsection{Improving KLARAPTOR: Finding optimal CUDA kernel parameters\hfill \normalsize\textnormal{Current}}
Improving the KLARAPTOR project: predicting optimal CUDA kernel parameters for fast GPU computation. My goals are:
\vspace{-0.5em}
\begin{itemize}
  \item Finding better performing CUDA kernel parameters by algorithmically generating better training data for KLARAPTOR
  \item Automatically detect kernel dimensions, variable relations, etc. used to better parallelize code via static analysis (on LLVM IR)
  % \item Working on rewrite using a \textbf{LR} parser with integrated lexer, a compiler to \textbf{LLVM} IR, and a strong typing system
        % TODO phrasing here sucks balls redo it
\end{itemize}
\vspace{-0.5em}
Speeds-up \textit{data-oblivious} computations (\textit{e.g.} matrix operations for faster machine learning/AI, computer graphics, \textit{etc.})

\section*{Projects}

\subsection{Experimental Language \normalsize\textnormal{- \href{https://github.com/ianayl/compiler}{repo}}\hfill \normalsize\textnormal{Current}}
A programming langauge written in \textbf{C++}, aiming to be a fast and modern replacement to system shells.
\vspace{-0.5em}
\begin{itemize}
  \item Devised an \textit{automatic} table-based \textbf{lexer generator}, and wrote a \textbf{recursive descent} parser following LL grammar
  \item Writing a \textbf{tree-walk evaluator} for interpretation, and passes to turn ASTs to CFGs to LLVM IR for compilation
  % \item Working on rewrite using a \textbf{LR} parser with integrated lexer, a compiler to \textbf{LLVM} IR, and a strong typing system
        % TODO phrasing here sucks balls redo it
  \item Static type system with type inference in works, goal for JIT compilation in future
\end{itemize}

\subsection{Mindless \normalsize\textnormal{- \href{https://github.com/ianayl/mindless}{repo}}\hfill \normalsize\textnormal{2021}}
    Proof-of-concept \textbf{unconditionally secure} biometrics password manager based on hashing algorithms and math
    \vspace{-0.5em}
\begin{itemize}
  \item Developed a password manager using facial recognition landmarks (extracted using \textbf{OpenCV}) with a web frontend (\textbf{Flask})
  \item Devised method to generate consistent passwords from facial landmark data: \textit{password is \textbf{never} stored! Not even encrypted}
    \item Finalist project in a 36-hour hackathon (Hackwestern 8), competing as a one-man team out of 346 teams of 4 to 5
\end{itemize}

\subsection{Rentura \textnormal{(Startup)} \hfill \normalsize\textnormal{2021 -- 2022}}
Lead the development and design of a \textit{minimum viable product} for a B2C furniture rental platform
\vspace{-0.5em}
\begin{itemize}
  \item Developed a backend in Javascript, implementing a \textbf{RESTful API} with \textbf{CRUD} endpoints secured with tokens (in \textbf{Express.js})
  \item Maintained \textbf{MongoDB} instances on a \textbf{Linux} server, and designed database schemas for an \textit{order management system }
  \item Developed a responsive, mobile-first \textit{e-commerce} frontend in \textbf{React.js} using \textbf{Sass} and \textbf{Axios} (\textbf{MERN} stack)
    % \item TODO add business stuff later
    % \item 2021\--Current
\end{itemize}

\subsection{Shell Site Generator (\texttt{shsg}) \normalsize\textnormal{- \href{https://github.com/ianayl/shsg}{repo}}\hfill \normalsize\textnormal{Current \textit{(On hiatus)}}}
  Ultra lightweight and portable static site generator that will run on literally \textit{anything} with a minimal UNIX-like enviroment
    % Easy to use static site generator that can be deployed on any minimal UNIX-like environment
\vspace{-0.5em}
\begin{itemize}
  \item Developing a static site generator using pure \textbf{POSIX \texttt{sh}} and \textbf{PCRE Regex}: \textit{No dependencies other than a UNIX shell + coreutils!}
  \item Transpiles markdown to themed webpages, but unlike current options (e.g. \textit{Jekyll}), no installation + alternate runtimes needed
    \item In works: parsing tables and lists in pure regex, removing dependency on PCRE (support for Busybox and other glorified toasters)
\end{itemize}

\section{Education}
\subsection{University of Western Ontario - \normalsize\textit{London, Canada} \hfill \normalsize\textnormal{2021 -- Current}}
\subsubsection{Candidate for Bachelor of Science} Specialist in Honours Computer Science, Anticipated in Spring 2025

\subsubsection{Awards} Dean's Honor List (2021-2022, 2022-2023), 3.9/4.0 GPA % \section{References}
% References available upon request

\end{document}
